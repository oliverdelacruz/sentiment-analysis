\section{Introduction}

Sentiment analysis is an intensely pursued research topic at the intersection of data science and computational linguistics. In general, sentiment analysis is understood as the deduction of an author's emotional characteristics on the basis of her text.

Research in sentiment analysis is fueled and facilitated by large amounts of available and easily usable data. One such source, which was used in this paper, are \textit{tweets}. Twitter \cite{twitter} provides large amounts of tweets labelled as either positive or negative, where the label is determined on the basis of emoticons in a tweet.

In this paper, we draw on 2.5 million such tweets to train a classification algorithm that aims at identifying positive and negative tweets, doing so in absence of emoticons. In advance of classification, we apply preprocessing and word-vector embedding methods to ease classification. The paper at hand also provides an evaluation of different such methods.